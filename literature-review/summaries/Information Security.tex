Nowadays, Information and Communication Technologies (ICT) an important success factor within any modern society. Many new technologies have emerged and have improved people lives and organizations’ services including governments, worldwide. The significant shift in the business environment, economic instability, and customers’ desires and expectations, increase the need to develop and adopt new IT innovations. Over the last decades, several strategic transformations in enterprises and the governmental sectors are based on ICT applications, which brought a lot of benefits. Consequently,the need for Information Security (InfoSec) becomes an essential matter as thousands of organizations worldwide are heavily dependent on information process systems to perform their daily tasks. Thus, it is a critical role to ensure that the information technology assets are secured and protected against IT threats.
Many scholars have defined "information security" from different perspectives as it includes multidimensional factors that are concentrated on the preservation and protection of information assets via the implementation of security technical,operational and physical controls \cite{Hamid2014,Posthumus2004}. While those controls  need to be improved, reviewed and monitored in regular bases to ensure that the organizations' business and security objectives are achieved\cite{ISO/IEC2014}. 

The national institute of Standard and Technology (NIST)\cite{Kissel2013} has defined information security as "the protection of information and information unauthorised access, use, disclosure, disruption, modification, or destruction in order to provide confidentiality, integrity, and availability ". \citet{Zafar2009}, they incorporate more components to InforSec definition in order gain a holistic view. These components are: establishing security policies and procedures, understanding and assessing potential security threats and risks, implementing and monitoring security controls, educating and training personnel in security awareness, performing permanent technology assessment and integrating information security governance. Information security mainly aims to preserve information confidentiality, unauthorised integrity and availability, which is known as (CIA) security triad\cite{pfleeger2007security}. \citet{ISO/IEC2014} also adds authenticity, accountability, non-repudiation and reliability . According to European Network and information Security Agency (ENISA)\cite{ENISA2006},information confidentiality means "The protection of communications or stored data against interception and reading by unauthorized persons" . Integrity is refereed to "The confirmation that data which has been sent, received, or stored are complete and unchanged." Availability is defined as "The fact that data is accessible and services are operational". 

