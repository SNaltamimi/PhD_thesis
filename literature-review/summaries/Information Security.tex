Nowadays, Information and Communication Technologies (ICT) an important success factor within any modern society. Many new technologies have emerged and have improved people lives and organizations’ services including governments, worldwide. The significant shift in the business environment, economic instability, and customers’ desires and expectations, increase the need to develop and adopt new IT innovations. Over the last decades, several strategic transformations in enterprises and the governmental sectors are based on ICT applications, which brought a lot of benefits. Consequently,the need for Information Security (InfoSec) becomes an essential matter as thousands of organizations worldwide are heavily dependent on information process systems to perform their daily tasks. Thus, it is a critical role to ensure that the information technology assets are secured and protected against IT threats.
Many scholars have defined "information security" from different perspectives as it includes multidimensional factors that are concentrated on the preservation and protection of information assets via the implementation of security technical,operational and physical controls \cite{Hamid2014,Posthumus2004}. While those controls  need to be improved, reviewed and monitored in regular bases to ensure that the organizations' business and security objectives are achieved\cite{ISO/IEC2014}. 


The national institute of Standard and Technology (NIST)\cite{Kissel2013} has defined information security as "the protection of information and information unauthorised access, use, disclosure, disruption, modification, or destruction in order to provide confidentiality, integrity, and availability ". \citet{Zafar2009}, they incorporate more components to InforSec definition in order gain a holistic view. These components are: establishing security policies and procedures, understanding and assessing potential security threats and risks, implementing and monitoring security controls, educating and training personnel in security awareness, performing permanent technology assessment and integrating information security governance. Information security mainly aims to preserve information confidentiality, unauthorised integrity and availability, which is known as (CIA) security triad\cite{pfleeger2007security}. \cite{ISO/IEC2014} also adds authenticity, accountability, non-repudiation and reliability . According to European Network and information Security Agency (ENISA)\cite{ENISA2006},information confidentiality means "The protection of communications or stored data against interception and reading by unauthorized persons" . Integrity is refereed to "The confirmation that data which has been sent, received, or stored are complete and unchanged." Availability is defined as "The fact that data is accessible and services are operational". 


% Health information systems security standards
The sensitive nature of health information over other personal information and the growth rate of dependency on health care information systems, have increased the need to robust information Security Management (ISM). If patients' information has been compromised, then the health organisation may suffer from lots of legal issues, which may turn to financial losses as well as a huge damage to the organisation reputation. Furthermore,the HIS now have been shifted from stand-alone system with specific end-users to includes patients at homes via the internet. This development in network and information exchange technologies have increased the type and capacity of the HIS threats and challenges.Such development in network and information exchange technologies have increased the type and capacity of the HIS threats and challenges \cite{Hsu}. 
In response to these security risks, initiatives from several countries and institutions have been launched to improve ISM practices, procedures and guidelines by developing many generic and specific security standards. These standards aim to help the organisations in several industries to utilise their resources and efforts efficiently in order to gain an adequate security level via the adoption of best security practices \cite{Rahim2016,Akowuah2013}.


\citet{Akowuah2013} in their literature survey have reviewed several security standards including NIST Special Publication 800-53 , HITRUST Common Security Framework (CSF), Control OBjective for Information and related Technology (COBIT), ISO/IEC27002:2005, ISO/IEC27001:2005, ISO27799:2008, ISO17090:2008,ISO/TS 25237:2008. \cite{Akowuah2013} aim was to facilitate the choosing process for a suitable security standard that can guide information security management practices in the healthcare industry.In this survey, many standards were reviewed and analysed in order to assist IT management in their initial steps toward security programs implementation. \citet{Akowuah2013} suggested that ISO 27799:2008 and its associated series ISO 17090:2008 and ISO/TS 25237:2008 were more suitable of all size organizations in the healthcare industry as they were tailored to handle various security aspects and technical issues within healthcare environment. Moreover, Health Information Trust Allianc (HITRUST) is a specific health security standard that can satisfy many big size organizations security needs. It requires a subscription with HITRUST to get an access for health information security materials and training courses.In the other hand, some security standards such as NIST SP 800-53, ISO 27002:2005, and COBIT were more generic standards that provide holistic security approaches and procedures. Thus, they can be used as a alternative reference during the implementation of  security programmes in the healthcare organizations\cite{Akowuah2013}. 

\subsection{Information Security Policy Compliance:}
TBD
\subsection{Information Security Challenges:}
\subsubsection{\textbf{Pivacy:}}

One of the most important security concerns of adopting the HIS applications such as EMR is the patient information privacy \cite{Mahfuth2016}.The sensitive nature of the patient health information and the widespread usage of EMR/EHR in healthcare organisations have increased the fears of security risks and vulnerabilities. Those security fears can originate from internal sources related to types of intentional and unintentional behaviour such as employee ignorance, curiosity, misuse of password, social engineering, etc. External threats may include intruder and hacker attempts, malicious software, spy ware and viruses attacks  \cite{narayana2010security}. In an effort to preserve the EHR integrity, confidentiality and availability from the potentials security and privacy breaches, many countries have conducted security laws and enforced compliance from all healthcare parties that store, process and exchange EHR electronically\cite{Hsu,narayana2010security,Rahim2016}. For instance, the Health Insurance Portability and Accountability Act of 1996 (“HIPAA”) in the USA, the Personal Information Protection and Electronic Documents Act (PIPEDA  Act) in Canada, the EU Cross-Border Health Care Directive 2011/24/EU1 and (Ley de Protección de Datos) law in Spain, etc \cite{Bensefia2014,U.S.DepartmentofHealthandHumanServices}. 


The aim of HIPPA, for example, is ensuring the confidentiality, availability and integrity of protected health information (PHI), while being stored, exchanged and processed by any formats (electronic, on the document or oral) between one or several healthcare providers. The PHI include individuals' mental and physical health history, health providers information including bills and any other information that can reveal patient identity \cite{andriole2014security,U.S.DepartmentofHealthandHumanServices}.
Moreover, the U.S. Department of Health and Human Services (“HHS”) produced the Standards for Privacy of Individually Identifiable Health Information (Privacy Rule) as a way to guide the actions during the implementation of (“HIPAA”). According to \cite{U.S.DepartmentofHealthandHumanServices}, the Pivacy Rule main objective " is to assure that individuals’ health information is properly protected while allowing the flow of health information needed to provide and promote high quality health care and to protect the public's health and well being". 

\citet{Mahfuth2016}a systematic literature review with objectives to examine the security and privacy concerns and challenges related to the Electronic Health Record Systems (EMRs) within the healthcare industry. Another objective was identifying and analysing the current security solutions to overcome the confidentiality violation concerns as a result of  EMRs adoption, which includes various range of security frameworks, controls, and policies. The findings showed that there was an increasing rate of EMRs adoption in developed and developing countries worldwide. Therefore, seeking and maintaining an optimal level of the EMR privacy against unauthorised access was a major security challenge for the healthcare team members, patients, IT experts and healthcare stakeholders. \citet{Mahfuth2016} argued that the developing countries had greater privacy risks regarding EMRs than the developed. This is due to poor IT experience and  infrastructure, the insufficient  security awareness level, and inadequate financial resources as well as the absence of security laws and regulations there. Moreover, the authors noted that the existing security solutions and policies were not sufficient to ensure a comprehensive protection of the EMR's health data privacy, which as a result may affect the healthcare Quality of Service (QoS) \cite{Mahfuth2016}.

\citet{bensefia2014proposed} proposed a novel EMRs privacy layered Architecture model. it aims to make a balance between maintaining the EMR privacy and at the same time ensuring EMR availability for other authorised health providers. As an effort in this direction, the  \cite{bensefia2014proposed} model encompasses of three main layers, administrative decisions, the hardware infrastructure and technological issues. The administrative decisions layer includes security rules, regulations and standards to be satisfied from all healthcare parties. The hardware infrastructure includes all the physical types of equipment that were involved in handling EMR. The last layer was technological issues, which was responsible to distinguish the EMR sensitive data set from the common EMR information and then placed it to private database with restricting security access controls.Thus, this private database and it sensitive EMR data will be accesses and shared via a proxy server that can grant IP addresses to authorized clients. 

