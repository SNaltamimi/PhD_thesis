Nowadays, Information and Communication Technologies (ICT) an important success factor within any modern society. Many new technologies have emerged and have improved people lives and organizations’ services including governments, worldwide. The significant shift in the business environment, economic instability, and customers’ desires and expectations, increase the need to develop and adopt new IT innovations. Over the last decades, several strategic transformations in enterprises and the governmental sectors are based on ICT applications, which brought a lot of benefits. Consequently,the need for Information Security (InfoSec) becomes an essential matter as thousands of organizations worldwide are heavily dependent on information process systems to perform their daily tasks. Thus, it is a critical role to ensure that the information technology assets are secured and protected against IT threats.
Many scholars have defined "information security" from different perspectives as it includes multidimensional factors that are concentrated on the preservation and protection of information assets via the implementation of security technical,operational and physical controls \cite{Hamid2014,Posthumus2004}. While those controls  need to be improved, reviewed and monitored in regular bases to ensure that the organizations' business and security objectives are achieved\cite{ISO/IEC2014}. 

The national institute of Standard and Technology (NIST)\cite{Kissel2013} has defined information security as "the protection of information and information unauthorised access, use, disclosure, disruption, modification, or destruction in order to provide confidentiality, integrity, and availability ". \citet{Zafar2009}, they incorporate more components to InforSec definition in order gain a holistic view. These components are: establishing security policies and procedures, understanding and assessing potential security threats and risks, implementing and monitoring security controls, educating and training personnel in security awareness, performing permanent technology assessment and integrating information security governance. Information security mainly aims to preserve information confidentiality, unauthorised integrity and availability, which is known as (CIA) security triad\cite{pfleeger2007security}. \citet{ISO/IEC2014} also adds authenticity, accountability, non-repudiation and reliability . According to European Network and information Security Agency (ENISA)\cite{ENISA2006},information confidentiality means "The protection of communications or stored data against interception and reading by unauthorized persons" . Integrity is refereed to "The confirmation that data which has been sent, received, or stored are complete and unchanged." Availability is defined as "The fact that data is accessible and services are operational". 

% Health information systems security standards
The sensitive nature of health information over other personal information and the growth rate of dependency on health care information systems, have increased the need to robust information Security Management (ISM). If patients' information has been compromised, then the health organisation may suffer from lots of legal issues, which may turn to financial losses as well as a huge damage to the organisation reputation. Furthermore,the HIS now have been shifted from stand-alone system with specific end-users to includes patients at homes via the internet. This development in network and information exchange technologies have increased the type and capacity of the HIS threats and challenges.Such development in network and information exchange technologies have increased the type and capacity of the HIS threats and challenges \cite{Hsu}. 
In response to these security risks, initiatives from several countries and institutions have been launched to improve ISM practices, procedures and guidelines by developing many generic and specific security standards. These standards aim to help the organisations in several industries to utilise their resources and efforts efficiently in order to gain an adequate security level via the adoption of best security practices \cite{Rahim2016,Akowuah2013}.

\citet{Akowuah2013} in their literature survey have reviewed several security standards including NIST Special Publication 800-53 , HITRUST Common Security Framework (CSF), Control OBjective for Information and related Technology (COBIT), ISO/IEC 27002:2005, ISO/IEC 27001:2005,ISO 27799:2008,ISO 17090:2008,ISO/TS 25237:2008. \cite{Akowuah2013} aim was to facilitate the choosing process for a suitable security standard that can guide information security management practices in the healthcare industry.In this survey, many standards were reviewed and analysed in order to assist IT management in their initial steps toward security programs implementation. \citet{Akowuah2013} suggested that ISO 27799:2008 and its associated series ISO 17090:2008 and ISO/TS 25237:2008 were more suitable of all size organizations in the healthcare industry as they were tailored to handle various security aspects and technical issues within healthcare environment. Moreover, Health Information Trust Allianc (HITRUST) is a specific health security standard that can satisfy many big size organizations security needs. It requires a subscription with HITRUST to get an access for health information security materials and training courses.In the other hand, some security standards such as NIST SP 800-53, ISO 27002:2005, and COBIT were more generic standards that provide holistic security approaches and procedures. Thus, they can be used as a alternative reference during the implementation of  security programmes in the healthcare organizations. 
