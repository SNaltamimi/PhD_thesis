The meaning of the term "Awareness", according to Cambridge dictionary, is "knowledge that something exists, or understanding of a situation or subject at the present time based on information or experience ".
In information security context, academics and practitioners have several perceptions about the definition information security awareness. \citet{Albrechtsen} stated that information security performance in any organization is affected by the overall awareness and behaviour of all users.   
NIST\cite{Kissel2013} defines Information Security Awareness(ISA) as "Activities which seek to focus an individual’s attention on an (information security) issue or set of issues" .  
Recently, many information security breaches have occurred at small, medium and large organisations in government and private sectors. Some of them due to vulnerabilities caused by Humans behaviour rather than  weaknesses in the IT controls. Many scholars argue that humans behaviour plays an important factor within information security domain, and poor behaviour intentionally or unintentionally can jeopardise all the security efforts \cite{Kruger2006,Butavicius2017,Bawazir2016,Giraldo2014,Lebek2013,McCormac2016,safa2015information,Benbasat2010a}.  

According to \citet{Parsons2013} study, the authors used web-based questionnaires and interviews within three government organisations in Australia. Three dimensions used Knowledge, Attitude and Behaviour (KAB) model to evaluate employee InfoSec awareness while using the work computer.To do that, eight InfoSec management area namely, the importance of InfoSec policies, rules of InfoSec polices, password management, the principle of InfoSec polices, email and internet usage, the consequence of behaviour and training. The results from a total sample of 203 employee responses revealed that the overall knowledge of InfoSec of the focused areas were very satisfactory and scored between 80 - 90 \% . Also, the employee attitude and behaviour toward InfoSec were overly good,but scored lower than knowledge. Furthermore, this result consistent with findings of the senior management interviews, which showed that the managers had a good understanding in general of their employee InfoSec weaknesses and concerns as well as the need for more InfoSec training to improve security awareness in reporting security incidents and the using of social websites and wireless technologies.
 

To get deeper insight to the affect of individual behaviour on the information security. Another study by \cite{Parsons2014} investigated factors influencing the employee accidental-naive behaviour while using a computer at work. The authors used self-reported behaviour as a measure to predict how an employee do behave and how they would behave. Those factors were an employee age , level of education, ability to control impulsively, familiarity with the computer along with the five personal traits.Via an online survey, a total sample of 500 Australian working adults was included in the final analysis. The results revealed that age, ability to control impassivity, conscientiousness, openness  and agreeableness have significantly and positivity influenced on self-reported accidental naive behaviour of employee. In another words, an employee accidental naive behaviour was likely to be more risky if the employee was younger and more impulsive and familiars with computers; less open, conscientious, and agreeable. The study suggested that addressing those factors can improve the overall security awareness and protection against  security risks related to the employee accidental naive behaviour.   
  
\citet{McCormac2016} conducted a survey of 505 working Australians at three governmental organisations. The aim of this study was to investigate the effect of human factors (individuals differences) on the ISA within seven security area including; mobile computing, information handling, incident reporting, password management, social media, internet use and email use. This study measured several humans' factors including individuals demographics (age and gender), personality characteristics (agreeableness, neuroticism, openness conscientiousness and extroversion), and risk taking propensity using Human Aspects of Information Security Questionnaire (HAIS-Q). Thus, Knowledge, Attitude and Behaviour (KAB) model was used to define the individuals ISA. Results showed that both age and gender had a small impact on the variance of individual security awareness; while most of the individuals' factors that  influenced significantly ISA were related to conscientiousness, agreeable, emotional stability and risk-taking propensity.

According to \citet{safa2015information}, several security breaches can occur as consequences of individual's ignorance, lack of awareness, resistance, apathy and negligence of security policies or procedures. 
These results are in alignment with  \citet{Hamid2014} study that aimed to assess postgraduate students awareness in a Malaysian's university (IIUM) against nine types of threats within information security domain, including Malware, Anti-virus, cloud computing, social engineering, password/identity theft, mobile security, firewall/ browser security, IS breach of social network "privacy/confidentiality" and online game. Findings revealed that no significant differences in ISA between Male and Female. Also, the results showed that the majority of respondent are conversant with information security issues and had an acceptable level of ISA. 

In the same direction, \citet{Grant2010} conducted a study that adopted \citet{Katz2005} efforts to assess the relationship between individuals behaviour and their security awareness via a research questionnaire. \citet{Grant2010} specifically investigated the effect of age, education, gender and occupation as demographic factors had on individuals security behaviours and practices. The findings indicated that security awareness level of male and older participants were less than female and younger counterparts. Participants with college educated were less security aware than other workers without a college education. Also, participants in non-technical jobs were more security aware than their colleagues in technical positions. 


 

