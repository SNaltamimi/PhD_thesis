 \subsubsection{\textbf{Neutralization Theory:}}
 
 \citet{Siponen2010}conducted an empirical study, and they argue that the employee intention to violation information security policies can be explained by neutralisation factors. The study proposed a theoretical model to test the impact of formal and informal sanctions along with shame on the employee violation intention and the effect of neutralisation factors to rationalise such behaviour. The finding revealed that the organization sanctions alone were not enough to decrease or prevent information policies violation intention as the employees tend to utilise neutralisation techniques to minimise the perceived harm of sanctions. Based on the findings,  authors suggested that the management should consider the importance of neutralisation factors during their efforts to develop and implement information security policies and security awareness champagnes.      

Further empirical study in Malaysia by \citet{Teh2015}, aimed to investigated the organizational factors that influence employees neutralization behaviour toward information security polices violations. The study model contains four main  factors include role conflict and role ambiguity from the security literature as well as tow factors drawn form social exchange theory, which were job satisfaction and organizational commitment. A total sample of 246 employees, working in nine banks, participated in this study. The results showed that role conflict had a positive influence on employees neutralization techniques toward ISP violation. In contrast , organization commitment, job satisfaction and role ambiguity had insignificant affect on neutralization techniques toward information security policy violation. 

Another empirical study by\citet{Kim2014} proposed an integrative behavioural model based on theory planned behaviour, protraction motivative theory and rational choice theory from social psychology science and neutralization theory from criminology field. The goal was to understand the behavioural factors that affected the employee actions toward compliance with organization information security polices. The authors used survey disseminated to 32 companies within 10 industries and a sample of 194 participants was collected. The results showed that neutralization techniques significantly weaken employee intention toward information security policies compliance. Also, response efficacy and normative beliefs positively impacted employees' intention to comply with the organization's security policies.

Another study conducted by \citet{Morris} examined digital piracy as an illegal behaviour via several theories applied in criminology field.  In particular, authors explored the role of Neutralisation techniques along with  Self-Control (SC), social learning (SL) and microanomie theories to understand both self-report and the willingness to engage in delinquent behaviour such as illegal sharing, copying or downloading of online products. 

 


  