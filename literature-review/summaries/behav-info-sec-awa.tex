 \subsubsection{\textbf{Neutralization Theory:}}
 
 \citet{Siponen2010}conducted an empirical study, they argue that the employee intention to violation security policies is influenced by neutralisation factors. The study proposed a theoretical model in order to test the impact of formal and informal sanctions along with shame on the employee violation intention and the effect of neutralisation factors to rationalise such behaviour. The finding reveals that the neutralisation techniques significantly impact the employees' policy violations intention as they been used by employees to decrease the perceived harm of sanctions if they were getting caught.Hence, informal sanction significantly affect the employees' violation intention in the existence of neutralisation,while formal sanctions effect found insignificant.  

Further empirical study in Malaysia by \citet{Teh2015}, aimed to investigated the organizational factors that influence employees neutralization behaviour toward information security polices violations. The study model contains four main  factors include role conflict and role ambiguity from the security literature as well as tow factors drawn form social exchange theory, which were job satisfaction and organizational commitment. A total sample of 246 employees, working in nine banks, participated in this study. The results showed that role conflict had a positive influence on employees neutralization techniques toward ISP violation. Furthermore, organization commitment, job satisfaction and role ambiguity had insignificant affect on neutralization techniques on information security policy violation context. 

Also, 


  