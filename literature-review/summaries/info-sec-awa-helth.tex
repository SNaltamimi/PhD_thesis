% Introduction About HIS

Healthcare industry is considered one of the most sophisticated businesses that interacts with a complex network of entities. Therefore, the use of information and communication technologies (ICT) in the healthcare sector has become imperative to support the activities of healthcare organisations. Hospitals often collect huge amounts of data to support their daily medical activities, as well as financial and managerial transactions, which have grown rapidly. Data at hospitals is generated from several sources, including patients, insurance companies, labs,pharmacy, etc. Thus, the management of such huge amounts of data requires an effective IT solutions that can satisfy many critical requirements such as easily accessible, cost-effective, reliable services and high quality.

% Definition and benefits of HIS

The need for technological advancements to improve healthcare services delivery, quality and performance have attracted the industry stakeholders to implement  several health information technologies. ISO 27799:2016 has defined Health Information Systems (HIS) as " \textit{a repository of information regarding the health of a subject of care in computer-processable form, stored and transmitted securely, and accessible by multiple authorised users}" \cite{TheInternationalOrganizationforStandardization2016}. Another definition of the HIS is based on the fact that it is a computer program, which  includes " a set of standards based on healthcare diagnosis, symptoms, cause, healthcare target and measurements" \cite{Pai2011}. The adoption of HIS have improved the compliance with the health care standards and disease control, which affect the overall quality delivery of healthcare services. Also, the implementation of clinical decision support tools have improved the diagnoses efficiency, which as a result have reduced significantly the total  rate and time of healthcare utilisation. \cite{chaudhry2006systematic,Akowuah2013}.

In this direction, several health information systems (HIS) have impacted positivity healthcare organisations such as E-Health, Electronic Mealth Records (EMR), Mobile health (mHealth),and Telemedicine, cloud computing in healthcare, big data analysis, health exchange and health sensing \cite{WorldHealthOrganization2016,Yang2015}. 


\subsection{Electronic Medical Records (EMR):}


The EMR replaced paper-based charts in hospitals and medical clinics to an electronic version that may allow the patient information to be integrated, transmitted, stored and shared in different systems and locations \cite{Rahim2016}. In many researches, Electronic Health Records(EHR) is knowen as Electronic Health Records(EHR)\cite{WorldHealthOrganization2016,Rahim2016,} or Personal Health Records (PHR).


The World Health Organization (WHO) \cite{WorldHealthOrganization2016} describes the EHRs as  " real-time, patient-centred records that provide immediate and secure information to authorized users. EHRs typically contain a patient’s medical history, diagnoses and treatment, medications, allergies, immunizations, as well as radiology images and laboratory results". 
 
 
The \citet{WorldHealthOrganization2013} (WHO) 2008 report stated that the implementation of the EMR, clinical decision tools and laboratory and pharmaceutical systems in poor African country such as Kenya, have reduced the practitioners errors and have enhanced both healthcare diagnoses and follow-up services. According to  \cite{OfHealth}, the implementation of EMR has provided healthcare organizations with significant advantages and can gain one or more of the following benefits:

- \textbf{Better quality of care}: the EMR has improved the concept of information exchanging between doctors, healthcare team members and departments as well as off-site health providers. As a result, the patient information can be accessed easily if a patient needs an emergency care or requires a specific medication. Like any computer system, the system administrators can make a full backup of the EMR, which can decrease the risk and cost of losing data if a disaster accrued  \cite{OfHealth}.

- \textbf{Improved care efficiency}: the EMRs is receiving data from different health information systems, so the patient information can be modified from different sources and locations. This means that the patient data are available to several health practitioners, and each of them can communicate through the EMR. Thus, it can give doctors a simple way to review the patient medical history or request a specific test or task from others. Such communications way can reduce the side effects of repeating some medical procedures such as X-rays as well as the time and cost associated with it \cite{OfHealth}. 

- \textbf{Improved care convenient}: the patient history can be exchanged and accessed easily, which are the basic principles of the EMR. So, no need of physical carriage for the paper records or filling more paper forms, which in return can reduce the waiting time for both the patients and doctors to receive or review the medical records\cite{OfHealth}.




















