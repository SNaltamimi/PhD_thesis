% Introduction About HIS

Healthcare industry is considered one of the most sophisticated businesses that interacts with a complex network of entities. Therefore, the use of information and communication technologies (ICT) in the healthcare sector has become imperative to support the activities of healthcare organisations. Hospitals often collect huge amounts of data to support their daily medical activities, as well as financial and managerial transactions, which have grown rapidly. Data at hospitals is generated from several sources, including patients, insurance companies, labs, etc. Thus, the management of such huge amounts of data requires an effective IT solutions that can satisfy many critical requirements, such as easily accessible, cost-effective, reliable services and high quality.The need for technological advancements to improve healthcare services delivery, quality and performance have attracted the industry stakeholders to implement  several Health Information Technologies (HIT).

ISO 27799:2008 has defined Health Information Systems (HIS) as " a repository of information regarding the health of a subject of care in computer-processable form, stored and transmitted securely, and accessible by multiple authorised users" \cite{ISO2008}. Health Information Systems include different technological applications such as E-Health, Electronic Health Records (EHR), Mobile health (mHealth),and Telemedicine, and so forth.

%Introduction about the EHR and its benefits

The EHR replaced the paper-based charts in the medical clinics to an electronic version.The World Health Organization (WHO) \cite{WorldHealthOrganization2016}  describes the EHRs as  " real-time, patient-centred records that provide immediate and secure information to authorized users. EHRs typically contain a patient’s medical history, diagnoses and treatment, medications, allergies, immunizations, as well as radiology images and laboratory results" 
 \cite{OfHealth}. In addition, many scholars named the EHRs as Electronic Medical Records(EMR).The implementation of EHR enhanced the health services in three main aspects \cite{chaudhry2006systematic,OfHealth}:


- Effect Quality: based on the literature, the HIS improve the healthcare services quality by improving compliance with the health care guidelines and disease surveillance as well as reducing prescription mistakes\cite{chaudhry2006systematic}. Also, EHR has improved data sharing and availability between doctors, healthcare team members and outside health providers in quick and easy manners \cite{OfHealth}.


- Effect Cost: much evidences have shown that the a significant reduction of time and cost associated with the routine duties performed by medical staff\cite{chaudhry2006systematic}.

- Effect Efficiency: the health services  have been enhanced by reducing the rate and time of healthcare utilisation by providing decision support tools with more accurate and consistent data, therefore, it will lead to better diagnoses \cite{chaudhry2006systematic}.

The EHR made the patient data available across several members of healthcare team . Thus, all the members are communicated through the system, which gave the doctor an easier way to view the patient medical history or asked another care member to perform specific test or task, therefore, it can reduce repetition of some medical procedures or the prescription of some wrong medications \cite{OfHealth}. 

The World Health organisation (WHO) \cite{WorldHealthOrganization2013} report stated that the implementation of the EMR, clinical decision tools and laboratory and pharmaceutical systems in poor African country such as Kenya, have reduced the practitioners errors and have enhanced both healthcare diagnoses and follow-up services.














