% Introduction About HIS

Healthcare industry is considered one of the most sophisticated businesses that interacts with a complex network of entities. Therefore, the use of information and communication technologies (ICT) in the healthcare sector has become imperative to support the activities of healthcare organisations. Hospitals often collect huge amounts of data to support their daily medical activities, as well as financial and managerial transactions, which have grown rapidly. Data at hospitals is generated from several sources, including patients, insurance companies, labs,pharmacy, etc. Thus, the management of such huge amounts of data requires an effective IT solutions that can satisfy many critical requirements such as easily accessible, cost-effective, reliable services and high quality.

% Definition and benefits of HIS

The need for technological advancements to improve healthcare services delivery, quality and performance have attracted the industry stakeholders to implement  several health information technologies. ISO 27799:2016 has defined Health Information Systems (HIS) as " \textit{a repository of information regarding the health of a subject of care in computer-processable form, stored and transmitted securely, and accessible by multiple authorised users}" \cite{TheInternationalOrganizationforStandardization2016}. Another definition of the HIS is based on the fact that it is a computer program, which  includes " a set of standards based on healthcare diagnosis, symptoms, cause, healthcare target and measurements" \cite{Pai2011}. The adoption of HIS have improved the compliance with the health care standards and disease control, which affect the overall quality delivery of healthcare services. Also, the implementation of clinical decision support tools have improved the diagnoses efficiency, which as a result have reduced significantly the total  rate and time of healthcare utilisation. \cite{chaudhry2006systematic,Akowuah2013}.Currently, many healthcare organizations are utilizing the HIS as a backbone of their operational services because it ability to be integrated with hospital clinical care and administrative systems\cite{simpson2006administrative,Rahim2016}   

In this direction, several health information systems (HIS) have impacted positivity healthcare organisations such as E-Health, Electronic Mealth Records (EMR), Mobile health (mHealth),and Telemedicine, cloud computing in healthcare, big data analysis, health exchange and health sensing \cite{WorldHealthOrganization2016,Yang2015}. 


\subsection{Electronic Medical Records (EMRs) systems:}
 
The  Electronic Medical Records(EMR) replaced paper-based charts in hospitals and medical clinics to an electronic version that may allow the patient information to be integrated, transmitted, stored and shared in different systems and locations \cite{Rahim2016}. It is difficult to find a stander definition for the EMR because the same meaning may refer to another term based on the perception of the EMR in a country or a healthcare sector.Thus,the EMR is considered a synonymous with abbreviations used elsewhere such as  Electronic Health Records(EHR)\cite{WorldHealthOrganization2016,Rahim2016,}, CPR (Computerized Patient Record) or Personal Health Records (PHR). In the other hand, several scholars have provided definitions that differentiated between those terms EMR, EHR and PHR \cite{Kierkegaard2011,Deutsch2010} . According to \cite{Yang2015,U.S.DepartmentofHealthandHumanServices2015}, three main differences of those terms  as the following:
\begin{itemize}
	\item\textbf{ Electronic Medical Record (EMR):} a health organisation is responsible to generate and control the EMR. Each EMR is a legal and digital record that includes all the patient medical history  during inpatient and outpatient visits. Basically, the EMR data are used for diagnoses purposes and shared locally within one health organization or institution\cite{Yang2015}. 
	\item \textbf{Electronic Health Record (EHR):} several health organisations are responsible to create,collect and maintain EHR data that are related to the patient healthcare.Thus, the EHR  may includes more comprehensive information as many sources contribute to it. Each EHR can be shared across different healthcare members , providers, regions, etc. When the EMR data are exchanged with external health organizations or entities, then they are considered EHR data and the EMR will be the main source of the transferred EHR\cite{Yang2015}.  
	\item \textbf{Personal Health Record (PHR)}: Each PHR record contains the same amount of EHR information, but the PHR data can be managed and accessed by individuals\cite{Yang2015,U.S.DepartmentofHealthandHumanServices2015}.
\end{itemize}

%\documentclass{article}
%\usepackage{graphicx}
%\graphicspath{ {images/} }

%\begin{document}
%	The universe is immense and it seems to be homogeneous, 
	%in a large scale, everywhere we look at.
%\begin{figure}[h]
%\centering
%\includegraphics[scale=1]{Distunction_between_EMR_EHR_PHR}
%\caption{Differences between EHR, EMR and PHR (Adapted from )}
%\end{figure}
%\end{document}

The World Health Organization (WHO) \cite{WorldHealthOrganization2016} describes the Electronic Health Record systems (EHRs) as  " real-time, patient-centred records that provide immediate and secure information to authorized users. EHRs typically contain a patient’s medical history, diagnoses and treatment, medications, allergies, immunizations, as well as radiology images and laboratory results". 
 
 
The \citet{WorldHealthOrganization2013} (WHO) 2008 report stated that the implementation of the clinical decision tools, laboratory and pharmaceutical systems in poor African country such as Kenya, have reduced the practitioners errors and have enhanced both healthcare diagnoses and follow-up services. According to  \cite{OfHealth}, the implementation of EMR has provided healthcare organizations with significant advantages and can gain one or more of the following benefits:
\begin{itemize}
\item \textbf{Better quality of care:} the EMR has improved the concept of information exchanging between doctors, healthcare team members and departments as well as off-site health providers. As a result, the patient information can be accessed easily if a patient needs an emergency care or requires a specific medication. Like any computer system, the system administrators can make a full backup of the EMR, which can decrease the risk and cost of losing data if a disaster accrued  \cite{OfHealth}.
\item \textbf{Improved care efficiency:} the EMRs is receiving data from different health information systems, so the patient information can be modified from different sources and locations. This means that the patient data are available to several health practitioners, and each of them can communicate through the EMR. Thus, it can give doctors a simple way to review the patient medical history or request a specific test or task from others. Such communications way can reduce the side effects of repeating some medical procedures such as X-rays as well as the time and cost associated with it \cite{OfHealth}. 

\item \textbf{Improved care convenient:} the patient history can be exchanged and accessed easily, which are the basic principles of the EMR. So, no need of physical carriage for the paper records or filling more paper forms, which in return can reduce the waiting time for both the patients and doctors to receive or review the medical records\cite{OfHealth}.
\end{itemize}


\subsection{Mobile Health (mHealth):}

TBD

\subsection{Cloud computing in Healthcare}

Cloud computing is one of the newest IT paradigms, which emerged in 2007, and it allows customers to use many advanced IT services and resources through the internet at the cloud service providers’ data centre \cite{Sultan2014}. According to \citet{Buyya2009}, the importance of cloud computing will increase and soon be considered a fifth utility after water, gas, electricity, and telephone. Using a pay-per-use model, several organisations can improve their business’ services and IT function’s performance, efficiency and quality, by paying for what they are using of the IT resources and services \cite{Abdollahzadehgan2013}. Virtualization, utility computing, grid computing and internet services are the fundamental concepts of the cloud computing evolution. Thus, cloud technology is considered a solution that is applying new forms of IT outsourcing \cite{tashkandi2015cloud,Chang2013}.

There is not a universal definition or clear description for cloud computing, and according to some studies, there are more than 22 definitions for it \cite{Sultan2014}. The ISO/IEC 17788 defines cloud computing as "Paradigm for enabling network access to a scalable and elastic pool of shareable physical
or virtual resources with self-service provisioning and administration on-demand". According to \cite{BSI2014}, Cloud computing consists of seven services categories: Communications as a Service (CaaS), Software-as-a-Service (SaaS), Platform-as-a-Service (PaaS), Network as a Service (NaaS), Compute as a Service (CompaaS), Data Storage as a Service (DSaaS)and Infrastructure-as-a-Service (IaaS). These categories required several cloud deployment models to organise, control and share the physical and virtual resources such as the following: 
\begin{itemize} 
	\item \textbf{public cloud}: it is a cloud deployment model, where the cloud service provider keeps the cloud services on premises and has full control on all physical and virtual resources related to the services. The cloud services in a public cloud can be shared and accessed from several cloud service customers and can be owned and managed by an organisation, business, academia, etc. The availability and accessibility of cloud services for the cloud services customers are subjected to few restrictions\cite{BSI2014}.
	
	\item \textbf{private cloud:} it is a cloud deployment model, where a single cloud service customer accesses the cloud services exclusively and has a control on all resources related to the services. The cloud services in a private cloud can be shared and accessed based on cloud service customer authorization and can be owned and managed by a third party or the customer itself. In private cloud, the availability and accessibility of cloud services are limited and subjected to wide range of restrictions\cite{BSI2014}. 
	
	\item \textbf{community cloud}: it is a cloud deployment model, where a small number of cloud service customers access the cloud services exclusively, and at least one of customers can control resources, and it may exist on or off premises. A community cloud can be owned, operated and managed by a third party or one or more customers in the cloud community. Unlike private cloud, the cloud service customers in the community cloud share the responsibilities and concerns of the service such as information security polices, procedures, requirements, etc\cite{BSI2014}. 
	
	\item \textbf{hybrid cloud:} it is a combination of two cloud deployment models, where appropriate technologies are in place to ensure the cloud services interoperability. A hybrid cloud may exist in or off premises and can be owned, operated and managed by a third party or an organisation itself\cite{BSI2014}.
	
\end{itemize}

In the healthcare industry, cloud computing has many promises. Due to the complexity of hospital information systems (HIS), cloud computing is a solution that opens a new horizon for patients’ records to be accessible via a secure authentication by authorised healthcare providers\cite{Grindle2013}.Also, it will help healthcare providers to gain important cost reductions and save money that is usually spent in buying and maintaining the needed hardware and software \cite{ahuja2012survey,Masrom2014}.

However, many organisations in different industries worldwide are still not ready to adopt cloud computing services. According to a recent report \cite{MarketsandMarkets}, the healthcare cloud computing market is predicted to grow nearly 9.48 billion dollars between 2015 to 2020, and only 4 percent of prospective cloud customers are healthcare organisations.The reasons for low adoption of cloud computing technology in healthcare industry are attributed to security and privacy fears, IT regulations compliance burdens, resource control and vendor lock-in concerns, as well as the lack of understanding of the technological, organisational and environmental factors that affect the decision makers when considering such technological shift \cite{MarketsandMarkets,tweel2012examining}


\subsection{Internet of Thing (IoT) in Healthcare}

TBD




















