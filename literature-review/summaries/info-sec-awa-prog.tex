Information security assurance requires suitable awareness training program that can guide users how to behave more secure manner \cite{safa2015information}. Successful ISA program can achieve a positive development of the individual security mindset from "become aware" to "be aware" to "stay aware" of of the security threats, polices,practices,etc \cite{Shaw2009,Schlienger2002}. \\
In this domain , \citet{merete2009effects} conducted an intervention study to measure the effectiveness of Individual Security  Awareness (ISA) training program using "e-learning software" in a maritime industry. This E-learning tool aims to enhance the employee's overall information security knowledge, awareness and behaviour through six training models each of which covers a security aspect. This study used two survey questionnaires, the first one was distributed a week before the ISA tool was launched, and the second survey was disseminated three weeks after the ISA training program was finished. The results reveal that the ISA e-learning program has improved the overall- short-term information security knowledge, awareness and behaviour of the employees in the intervention group. The employees who were finished the training program showed a significant awareness enhancement related to security concerns. For instance, the ISA participants showed more tendency to report frequent security incident violations, secure their unattended personal computer, and protect their password. These results identify that the improvement of security awareness has a positive effect on the employee's security behaviour.

\citet{Katz2005} has evaluated the need for information security awareness educational programs for university employees and faculty members. The research findings showed that most employees have proper behaviour about the information security fundamentals. For instance, most participants agree with the statement that asserts to turn off their personal computer before they left their offices at night. Also, the majority of respondents have a high level of password security and they never opened email attachments from an unknown sender. In contrast, more than 50 percent of the participants never read the university security policies. Also, there was a lack of some soft technical skills in ISA contexts such as performing Anti-Virus scan and sensitive data back up on regular basis. Thus, the author suggested several ISA programmes on the individual level to overcome the current insider issues in their information security practices.
\\
\citet{Shaw2009} conducted a laboratory experiment to investigate the effectiveness of online media richness as a learning and training technique to increase the information security awareness perception, comprehension and projection of security threats. A sample of 153 freshmen students participating in three training sessions, each contained 51 students. Those sessions were assigned to implement three types of media namely; hypermedia, multimedia and hypertext. Six hypothesis were tested based on the idea that each of those media can provide various information richness; as a result, the impact on the effectiveness of ISA level would vary. The results validate that security perception had a positive affect on both security comprehension and projection. Security perception and comprehension both can significantly advance the user capability of risk projection. Moreover, due to the differences in information richness, the hybemedia based ISA program was the most effective learning media to improve the security awareness perception, comprehension and projection. Also, students who were receiving multimedia based training exceed hypertext based in security comprehension and projection. The study suggested that the ISA program stockholders should balance between the targeted audience and the level of information richness of the ISA program to maintain an appropriate learning performance.
  




